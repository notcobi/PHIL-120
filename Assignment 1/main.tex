%%% Template originaly created by Karol Kozioł (mail@karol-koziol.net) and modified for ShareLaTeX use

\documentclass[a4paper,11pt]{article}

\usepackage[T1]{fontenc}
\usepackage[utf8]{inputenc}
\usepackage{graphicx}
\usepackage{xcolor}
\usepackage{lmodern}
\usepackage{framed}

\usepackage{tgtermes}

\usepackage[
pdftitle={PHIL 120 Assignment}, 
pdfauthor={Alex Diep, University of Alberta},
colorlinks=true,linkcolor=blue,urlcolor=blue,citecolor=blue,bookmarks=true,
bookmarksopenlevel=2]{hyperref}
\usepackage{amsmath,amssymb,amsthm,textcomp}
\usepackage{enumerate}
\usepackage{multicol}
\usepackage{tikz}
\usepackage{enumitem} % Add this line to include the enumitem package


\usepackage{geometry}
\geometry{left=25mm,right=25mm,%
bindingoffset=0mm, top=20mm,bottom=20mm}


\linespread{1.3}

\newcommand{\linia}{\rule{\linewidth}{0.5pt}}

% custom theorems if needed
\newtheoremstyle{mytheor}
    {1ex}{1ex}{\normalfont}{0pt}{\scshape}{.}{1ex}
    {{\thmname{#1 }}{\thmnumber{#2}}{\thmnote{ (#3)}}}

\theoremstyle{mytheor}
\newtheorem{defi}{Definition}

% my own titles
\makeatletter
\renewcommand{\maketitle}{
\begin{center}
\vspace{2ex}
{\huge \textsc{\@title}}
\vspace{1ex}
\\
\linia\\
\@author \hfill \@date
\vspace{4ex}
\end{center}
}
\makeatother
%%%



% custom footers and headers
\usepackage{fancyhdr,lastpage}
\pagestyle{fancy}
\lhead{}
\chead{}
\rhead{}
\lfoot{Assignment \textnumero{} 1}
\cfoot{Alex Diep}
\rfoot{Page \thepage\ /\ \pageref*{LastPage}}
\renewcommand{\headrulewidth}{0pt}
\renewcommand{\footrulewidth}{0pt}

% Define the custom page style for the first page
\fancypagestyle{firstpage}{
  \fancyhf{} % Clear all header and footer fields
  \fancyfoot[L]{Assignment \textnumero{} 1}
  \fancyfoot[C]{}
  \fancyfoot[R]{Page \thepage\ /\ \pageref*{LastPage}}
  \renewcommand{\headrulewidth}{0pt} % Remove header rule
  \renewcommand{\footrulewidth}{0pt} % Remove footer rule
}

\let\biconditional\leftrightarrow

%


%%%----------%%%----------%%%----------%%%----------%%%

\begin{document}

\title{PHIL 120 Assignment 1 -- Symbolic Logic I}

\author{Alex Diep, University of Alberta}

\date{\today}

\maketitle
% \thispagestyle{firstpage}

\paragraph{I:} \textit{Symbolize the following English sentences into TFL, using the following symbolization key. (0.5 point each)}
\begin{enumerate}[label=\Alph*:]
    \item Adam goes/will go to the hockey game.
    \item Betty goes/will go to the hockey game.
    \item Carol goes/will go to the hockey game.
    \item Daniel goes/will go to the hockey game.
\end{enumerate}
\textbf{1.} \textit{Adam and Carol will go to the hockey game.} \\
$\boxed{(A \land C)}$ \\[1em]
\textbf{2.} \textit{Not both Adam and Betty are going to the hockey game if Carol is going to the hockey game.} \\
$\boxed{[C \to \neg (A \land B)]}$ \\[1em]
\textbf{3.} \textit{Adam goes to the hockey game if and only if Carol goes to the hockey game.} \\
$\boxed{(A \biconditional C)}$ \\[1em]
\textbf{4.} \textit{Either Carol goes to the hockey game or Daniel goes to the hockey game, but both Adam and Betty go to the hockey game only if Carol goes to the hockey game.} \\
$\boxed{[(C \lor D) \land ((A \land B) \to C)]}$ \\[1em]
\textbf{5.} \textit{Someone between Adam, Betty, Carol, and Daniel will go to the hockey game, but not all of them.} \\
$\boxed{[(A \lor B \lor C \lor D) \land \neg (A \land B \land C \land D)]}$ \\[1em]

\paragraph{II:} \textit{In each of the following three expressions, determine whether it is a sentence of TFL or not. (0.25 point each) Explain why it is or it is not. (0.25 point each)} \\[1em]
\textbf{6.} $[G \to (\neg \neg L \lor (D \lor \neg C))] \biconditional [(C \biconditional (H \land \neg D)) \biconditional R]$ \\
Let us examine the LHS of the biconditional. The LHS is a sentence of TFL since
\begin{itemize}
    \item $G, L, D, C$ are atomic sentences.
    \item $\neg \neg L = L$ is a double negation sentence.
    \item $\mathbb{X} = (D \lor \neg C)$ is an inclusive disjunction sentence.
    \item $\mathbb{Y} = (\neg \neg L \lor \mathbb{X})$ is an inclusive disjunction sentence.
    \item $(G \to \mathbb{Y})$ is a conditional sentence.
\end{itemize}
Let us examine the RHS of the biconditional. The RHS is a sentence of TFL since
\begin{itemize}
    \item $C, H, D, R$ are atomic sentences.
    \item $\neg D$ is a negation sentence.
    \item $\mathbb{Z} = (H \land \neg D)$ is a conjunction sentence.
    \item $\mathbb{W} = (C \biconditional \mathbb{Z})$ is a biconditional sentence.
    \item $(\mathbb{W} \biconditional R)$ is a biconditional sentence.
\end{itemize}
\begin{framed}
\noindent Since both the LHS and RHS of the biconditional are sentences of TFL, the entire expression is a sentence of TFL. 
\end{framed}

\noindent\textbf{7.} $[(G \to (\neg \neg L \lor (D \lor \neg C))) \biconditional C] \biconditional [(C \land \neg D) \biconditional R]$ \\
Let us examine the LHS of the biconditional. The LHS is a sentence of TFL since
\begin{itemize}
    \item $G, L, D, C$ are atomic sentences.
    \item $\mathbb{X} = (G \to (\neg \neg L \lor (D \lor \neg C)))$ was established as a sentence of TFL in question 6.
    \item $(\mathbb{X} \biconditional C)$ is a biconditional sentence.
\end{itemize}
Let us examine the RHS of the biconditional. The RHS is a sentence of TFL since
\begin{itemize}
    \item $C, D, R$ are atomic sentences.
    \item $\neg D$ is a negation sentence.
    \item $\mathbb{Z} = (C \land \neg D)$ is a conjunction sentence.
    \item $(\mathbb{Z} \biconditional R)$ is a biconditional sentence.
\end{itemize}
\begin{framed}
\noindent Since both the LHS and RHS of the biconditional are sentences of TFL, the entire expression is a sentence of TFL.
\end{framed}

\noindent\textbf{8.} $(G \to \neg (\neg L \lor (D \lor \neg C))) \biconditional ((C \biconditional (H \land \neg D)) \biconditional R)$ \\
The RHS is a sentence since it is equivalent to the RHS of the biconditional in question 6. \\

\noindent Let us examine the LHS of the biconditional. The LHS is a sentence of TFL since
\begin{itemize}
    \item $G, L, D, C$ are atomic sentences.
    \item $\neg L$ is a negation sentence.
    \item $\neg C$ is a negation sentence.
    \item $\mathbb{X} = (D \lor \neg C)$ is an inclusive disjunction sentence.
    \item $\mathbb{Y} = (\neg L \lor \mathbb{X})$ is an inclusive disjunction sentence.
    \item $\neg \mathbb{Y}$ is a negation sentence.
    \item $(G \to \neg \mathbb{Y})$ is a conditional sentence.
\end{itemize}
\begin{framed}
\noindent Since both the LHS and RHS of the biconditional are sentences of TFL, the entire expression is a sentence of TFL.
\end{framed}

\paragraph{III:} \textit{In each of the following two sentences, find the main connective. (0.5 point each)} \\[1em]
\noindent\textbf{9.} $[((E \biconditional (\neg B \lor (D \to \neg C))) \lor C) \to ((C \land \neg D) \biconditional A)]$ \\
Using the counting method,
\begin{align*}
    \underbrace{[((}_{+3}E \biconditional \underbrace{(}_{+1}\neg B \lor \underbrace{(}_{+1}D \to \neg C\underbrace{)))}_{-3} \lor C\underbrace{)}_{-1} \underbrace{\to}_{=1} ((C \land \neg D) \biconditional A)]
\end{align*}
\begin{framed}
\noindent Therefore, the main connective is the conditional connective ${\to}$.
\end{framed}

\noindent\textbf{10.} $\neg((E \biconditional \neg(\neg \neg B \lor D)) \land (\neg(\neg C \to C) \to ((C \land \neg D) \biconditional A)))$ 
\begin{framed}
\noindent Since the sentence is in the form $\neg(\mathbb{X})$, the main connective is the negation connective ${\neg}$.
\end{framed}
\end{document}
