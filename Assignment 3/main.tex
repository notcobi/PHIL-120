%%% Template originaly created by Karol Kozioł (mail@karol-koziol.net) and modified for ShareLaTeX use

\documentclass[letterpaper,11pt]{article}

\usepackage[T1]{fontenc}
\usepackage[utf8]{inputenc}
\usepackage{graphicx}
\usepackage{xcolor}
\usepackage{lmodern}
\usepackage{framed}
\usepackage{tgtermes}
\usepackage[arrayenv=tabular]{fitch}
\renewenvironment{fitchproof}
  {\noindent\par\noindent\begin{nd}}
  {\end{nd}\par\noindent\ignorespacesafterend}
% \def\ndrules{
%   \def\bi{\by{{\eiff}I}}%
%   \def\be{\by{{\eiff}E}}%
  \def\ci{\by{{$\rightarrow$}I}}%
  \def\ce{\by{{$\rightarrow$}E}}%
  \def\mp{\by{MP}}% TB: modus ponens; same as conditional elimination
%   \def\Ai{\by{$\forall$I}}%
%   \def\Ae{\by{$\forall$E}}%
%   \def\Ei{\by{$\exists$I}}%
%   \def\Ee{\by{$\exists$E}}%
  \def\ae{\by{{$\land$}E}}% and elimination
  \def\ai{\by{{$\land$}I}}% and introduction
  \def\oi{\by{{$\lor$}I}}% or introduction
%   \def\oe{\by{{\eor}E}}%
%   \def\ni{\by{{\enot}I}}%
%   \def\ne{\by{{\enot}E}}%
    \def\ne{\by{{$\neg$}E}}% negation elimination
%   \def\ri{\by{{\enot}E}}% RZ: this is now \enot E
%   \def\re{\by{X}}% RZ: this is now X (explosion)
%   \def\ii{\by{$=$I}}%
%   \def\ie{\by{$=$E}}%
%   \def\tnd{\by{LEM}}% RZ: Law of excluded middle
  \def\ip{\by{IP}}% RZ: indirect proof
%   \def\dne{\by{DNE}}% TB: double negation elimination (derived rule)
  \def\mt{\by{MT}}% TB: modus tollens (derived rule)
  \def\ds{\by{DS}}% TB: disjunctive syllogism (a derived rule in Cambridge version)
%   \def\dem{\by{DeM}}% TB: De Morgan rule (derived rule)
%   \def\cq{\by{CQ}}% TB: conversion of quantifiers (derived rule)
%   \def\boxi{\by{{\ebox}I}}%
%   \def\boxe{\by{{\ebox}E}}%
%   \def\mc{\by{MC}}%
%   \def\diadf{\by{Def{\ediamond}}}%
%   \def\rt{\by{R$\mathbf{T}$}}%
%   \def\rfour{\by{R$\mathbf{4}$}}%
%   \def\rfive{\by{R$\mathbf{5}$}}%
%   \def\ellipsesline{\have[ ]{}{\vdots}}%
% }
% \def\by#1#2{#1} % allow \by outside proofs

\usepackage[
pdftitle={PHIL 120 Assignment}, 
pdfauthor={Alex Diep, University of Alberta},
colorlinks=true,linkcolor=blue,urlcolor=blue,citecolor=blue,bookmarks=true,
bookmarksopenlevel=2]{hyperref}
\usepackage{amsmath,amssymb,amsthm,textcomp}
\usepackage{enumerate}
\usepackage{multicol}
\usepackage{tikz}
\usepackage{enumitem} % Add this line to include the enumitem package
\usepackage{booktabs}

\usepackage{geometry}
\geometry{left=25mm,right=25mm,%
bindingoffset=0mm, top=20mm,bottom=20mm}


\linespread{1.3}

\newcommand{\linia}{\rule{\linewidth}{0.5pt}}

% custom theorems if needed
\newtheoremstyle{mytheor}
    {1ex}{1ex}{\normalfont}{0pt}{\scshape}{.}{1ex}
    {{\thmname{#1 }}{\thmnumber{#2}}{\thmnote{ (#3)}}}

\theoremstyle{mytheor}
\newtheorem{defi}{Definition}

% my own titles
\makeatletter
\renewcommand{\maketitle}{
\begin{center}
\vspace{2ex}
{\huge \textsc{\@title}}
\vspace{1ex}
\\
\linia\\
\@author \hfill \@date
\vspace{4ex}
\end{center}
}
\makeatother
%%%



% custom footers and headers
\usepackage{fancyhdr,lastpage}
\pagestyle{fancy}
\lhead{}
\chead{}
\rhead{}
\lfoot{Assignment \textnumero{} 3}
\cfoot{Alex Diep}
\rfoot{Page \thepage\ /\ \pageref*{LastPage}}
\renewcommand{\headrulewidth}{0pt}
\renewcommand{\footrulewidth}{0pt}

% Define the custom page style for the first page
\fancypagestyle{firstpage}{
  \fancyhf{} % Clear all header and footer fields
  \fancyfoot[L]{Assignment \textnumero{} 1}
  \fancyfoot[C]{}
  \fancyfoot[R]{Page \thepage\ /\ \pageref*{LastPage}}
  \renewcommand{\headrulewidth}{0pt} % Remove header rule
  \renewcommand{\footrulewidth}{0pt} % Remove footer rule
}

\let\biconditional\leftrightarrow

%


%%%----------%%%----------%%%----------%%%----------%%%

\begin{document}

\title{PHIL 120 Assignment 3 -- Symbolic Logic I}

\author{Alex Diep, University of Alberta}

\date{\today}

\maketitle

\paragraph{Question 1: Is the following an expression of FOL? (0.5 points)}
\begin{equation*}
    Q(x,y) \rightarrow \lor B(x)
\end{equation*}
\begin{framed}
No, the $\lor$ and $\rightarrow$ don't correctly act as quantifiers in this expression.
\end{framed}

\paragraph{Question 2: What are the scopes of the two quantifiers in the following sentence? (0.5 points)}
\begin{equation*}
    \forall x \left( Q(p) \leftrightarrow \left( A(x,c) \land \exists y \left( R(y) \land Q(x) \right) \right) \right) \rightarrow T(d,a)
\end{equation*}
\begin{framed}
    \begin{equation*}
        \underbrace{\forall x ( Q(p) \leftrightarrow ( A(x,c) \land \underbrace{\exists y \left( R(y) \land Q(x) \right)}_{\forall y} ))}_{\forall x} \rightarrow T(d,a)
    \end{equation*}
\end{framed}

\paragraph{Question 3: Which are, if any, the free variables in the following formula of FOL? (0.5 points)}
\begin{equation*}
    G(b,y) \rightarrow \exists x F(a,x)
\end{equation*}
\begin{framed}
    The free variables are $b$, $y$, and $a$.  
\end{framed}

\paragraph{Question 4: Is the following a formula of FOL? (0.5 points)}
\begin{equation*}
    \exists y \left( Q(y) \land R(y) \right)
\end{equation*}
\begin{framed}
    Yes, specifically a sentence of FOL.
\end{framed}

\paragraph{Question 5: Is the following a sentence of FOL? (0.5 points)}
\begin{equation*}
    \forall y A(y) \leftrightarrow \exists x B(a,x)
\end{equation*}
\begin{framed}
    No, the variable $a$ is free.
\end{framed}

For the next 5 questions (6-10):
\begin{enumerate}
    \item Provide a symbolization key (one key for all sentences)
    \item A domain
    \item Symbolize the following English sentences into FOL
\end{enumerate}
\paragraph{Question 6: ``Not all movies are artsy'' (0.5 points)}
\phantom{a}
\begin{framed}
    \begin{itemize}
        \item Domain: Movies
        \item Symbolization Key:
        \begin{itemize}
            \item $M(x)$: $x$ is a movie
            \item $A(x)$: $x$ is artsy
        \end{itemize}
        \item $\neg \forall x \left(M(x) \rightarrow A(x)\right)$
    \end{itemize}
\end{framed}

\paragraph{Question 7: ``All teachers like some movies'' (0.5 points)}
\phantom{a}
\begin{framed}
    \begin{itemize}
        \item Domain: People
        \item Symbolization Key:
        \begin{itemize}
            \item $T(x)$: $x$ is a teacher
            \item $L(x,y)$: $x$ likes $y$
            \item $M(x)$: $x$ is a movie
        \end{itemize}
        \item $\forall x \left(T(x) \rightarrow \exists y \left(M(y) \land L(x,y)\right)\right)$
    \end{itemize}
\end{framed}

\paragraph{Question 8: ``Some artsy movies are boring, and some boring teacher is artsy'' (0.5 points)}
\phantom{a}
\begin{framed}
    \begin{itemize}
        \item Domain: Movies and People
        \item Symbolization Key:
        \begin{itemize}
            \item $M(x)$: $x$ is a movie
            \item $A(x)$: $x$ is artsy
            \item $B(x)$: $x$ is boring
            \item $T(x)$: $x$ is a teacher
        \end{itemize}
        \item $\exists x \left(\left(M(x) \land A(x)\right) \land B(x)\right) \land \exists y \left( \left(T(y) \land B(y)\right) \land A(y)\right)$
    \end{itemize}
\end{framed}

\paragraph{Question 9: ``Only Amy likes artsy movies'' (0.5 points)}
\phantom{a}
\begin{framed}
    \begin{itemize}
        \item Domain: People and Movies
        \item Symbolization Key:
        \begin{itemize}
            \item $a$: Amy
            \item $M(x)$: $x$ is a movie
            \item $L(x,y)$: $x$ likes $y$
            \item $A(x)$: $x$ is artsy
        \end{itemize}
        \item $\forall x \forall y \left( \left( \left(M(y) \land L(x, y)\right) \land A(y) \right) \rightarrow (x = a)\right)$
    \end{itemize}
\end{framed}

\paragraph{Question 10: ``At least one artsy movie is not boring'' (0.5 points)}
\phantom{a}
\begin{framed}
    \begin{itemize}
        \item Domain: Movies
        \item Symbolization Key:
        \begin{itemize}
            \item $M(x)$: $x$ is a movie
            \item $A(x)$: $x$ is artsy
            \item $B(x)$: $x$ is boring
        \end{itemize}
        \item $\exists x \left( \left(M(x) \land A(x) \right) \land \neg B(x)\right)$
    \end{itemize}
\end{framed}
\end{document}
